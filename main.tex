%Preamble
% noindent
\documentclass[12pt,a4paper]{article}       % Declares document size

\usepackage[
  margin=20mm,
  paperwidth=21cm,
  paperheight=180cm
]{geometry}          % Defines margins
\usepackage[utf8]{inputenc}                 % Defines encription
\usepackage[magyar]{babel}                  % Defines language
\usepackage{t1enc}                          % Correctly hyphenates vowels

% \usepackage{tikz}                           % Tikz and PGF for graphics
% \usepackage{pgfplots}

\usepackage{xfrac}                          % Slanted fractions
\usepackage{lastpage}                       % Reference lastpage in footer
\usepackage{amsmath}                        % Math library
\usepackage{amssymb}                        % Symbols library
\usepackage{physics}                        % Physics library
\usepackage{subfiles}                       % Use subfiles
\usepackage{float}                          % Float
\usepackage{icomma}                         % Intelligent decimal separator
\usepackage{tabto}                          % Tab to a specific location in a document
\usepackage{multicol}                       % User nultiple cols
\usepackage{hyperref}                       % Mark refs
\usepackage[                                % More math font
  cal=rsfso, bb=ams
]{mathalpha}

\usepackage{fancyhdr}                       % Fancy design for pages
\renewcommand{\headrulewidth}{0.5pt}
\renewcommand{\footrulewidth}{0.5pt}
\setlength{\headheight}{20pt}
\pagestyle{fancy}
\cfoot{\thepage\ / \pageref{LastPage}}
\lhead{Polimertechnika minimumkérdések}
\rhead{\emph{Készítette:} Sándor Tibor}

% \numberwithin{equation}{section}            % Numbering and indenting
% \numberwithin{figure}{section}
% \numberwithin{table}{section}
\setlength{\parindent}{0mm}
\setlength{\parskip}{.66em}

\usepackage{xparse}                         % Conditional commands
\usepackage{tcolorbox}                      % Colored boxes
\tcbuselibrary{breakable, skins, listings, raster}
\newtcolorbox{cbox}[2]{
  shadow={0mm}{0mm}{-1.5mm}{fill=yellow!75!red,opacity=0.5},
  enhanced,
  colback=red!5!white,
  colframe=red!75!black,
  title={#2},
}

\usepackage{chemfig}
\usepackage{forest}
\usepackage{qtree}

\title{Polimertechnika minimumkérdések \\ \large(BMEGEPTBM01) \\ \texttt{v1.1.3}}
\author{Sándor Tibor}


% indent

\begin{document}
% \begin{center}
%   \Large \textbf{Polimertechnika minimumkérdések}
% \end{center}

\maketitle
\thispagestyle{fancy}

\begin{tcbitemize}[
    % sharp corners,
    % shadow={0mm}{0mm}{-1mm}{fill=yellow!75!red,opacity=0.5},
    enhanced, raster columns=1, raster row skip=1em, raster equal height=rows,
    colback=red!20!yellow!5!white, colframe=yellow!25!black, breakable,
    enhanced jigsaw
  ]

  \tcbitem[title={\# \thetcbrasternum –
        Mi a polimer?
      }]
  A polimer molekula olyan \textbf{nagyméretű molekula} (makromolekula),
  amelyet nagyon sok (poli) láncszerűen összekapcsolt \textbf{ismétlődő egység}
  (mer) alkot. A gyakorlatban ez több száz, általában minimum ezer ismétlődő
  egység összekapcsolását jelenti, melyek \textbf{kovalens kötéssel}
  kapcsolódnak egymáshoz.



  \tcbitem[title={\# \thetcbrasternum –
        Mi a polimertechnika?
      }]
  A polimertechnika (polymer engineering) minden olyan \textbf{műszaki
    tevékenység}, amelyet polimerekkel végzünk. A polimertechnika körébe
  tartozik:
  \begin{multicols}{3}
    \begin{itemize}
      \item előállítás,
      \item anyagtudomány,
      \item vizsgálat,
      \item módosítás,
      \item feldolgozás,
      \item műszaki feladatok.
    \end{itemize}
  \end{multicols}



  \tcbitem[title={\# \thetcbrasternum{} –
        Mi a monomer?
      }]
  A monomer \textbf{polimerizációra alkalmas kisméretű mulekula}. Kovalens
  kötésekkel kapcsolódnak össze ismétlődő egységekké.



  \tcbitem[title={\# \thetcbrasternum{} –
        Mit jelent az ismétlődő egység?
      }]
  Az ismétlődő egységek, polimerizációra alkalmas kisméretű molekulákból, ún.
  \textbf{momomerekből} származtatottak és \textbf{kovalens kötésekkel}
  kapcsolódnak össze.



  \tcbitem[title={\# \thetcbrasternum{} –
        Mi a kompaund?
      }]
  Adott célra előállított \textbf{keverék}, keveréssel állítjuk őket elő.
  \begin{center}
    \Tree[.{Keverés}
        [.{Száraz keverés\\(szilárd)}
            [.Szakaszos
                {Buktatott hordó}
            ]
            [.Folyamatos
                {Vándorcsigával ellátott\\kúpos siló}
            ]
        ]
        [.{Nedves keverés\\(folyadék)}
            [.Szakaszos
                {Hengerszék}
                {Belső\\keverő}
            ]
            [.Folyamatos
                {Extrúder}
                {Statikus\\keverő}
            ]
        ]
    ]
  \end{center}
  % \begin{multicols}{2}
  %   \begin{itemize}
  %     \item Száraz keverés (szilárd)
  %           \begin{itemize}
  %             \item szakaszos
  %                   \begin{itemize}
  %                     \item buktatott hordó
  %                   \end{itemize}
  %             \item folyamatos
  %                   \begin{itemize}
  %                     \item vándorcsigával ellátott kúpos siló
  %                   \end{itemize}
  %           \end{itemize}
  %           \vfill\null\columnbreak
  %     \item Nedves keverés (folyadék)
  %           \begin{itemize}
  %             \item szakaszos
  %                   \begin{itemize}
  %                     \item hengerszék
  %                     \item belső keverő
  %                   \end{itemize}
  %             \item folyamatos
  %                   \begin{itemize}
  %                     \item extrúzió
  %                     \item statikus keverő
  %                   \end{itemize}
  %           \end{itemize}
  %   \end{itemize}
  % \end{multicols}


  Lehet továbbá \textbf{disztributív} (komponensek méretcsökkenésével nem járó,
  eloszlató, extenzív keverés) és \textbf{diszperzív} (az összetartozó
  komponensek méretcsökkenésével járó, intenzív keverés)



  \tcbitem[title={\# \thetcbrasternum{} –
        Mi a polimer blend?
      }]
  A blend egy polimer-polimer \textbf{keverék}. Csak \textbf{fizikai} kapcsolat
  alakul ki, kémiai nem. Az anyagokat melegen összekeverjük, majd lehűtjük. Az
  anyagoknak kompatibilisnek (összeférhetőnek) kell lenni egymással, hogy blend
  legyen készíthető. Pl. PC–ABS.



  \tcbitem[title={\# \thetcbrasternum{} –
        Mi a kopolimer?
      }]
  Többféle ismétlődő egységet tartalmazó polimer, pl.: ABS. Típusai:

  \definesubmol\cA{\textcolor{red!50!yellow!80!black}{A}}
  \definesubmol\cB{\textcolor{yellow!75!black}{B}}

  % noindent
  \begin{table}[H]
    \centering
    \begin{tabular}{l l r}
      Alternáló             & \chemfig[atom sep=2em]{
        \phantom{X}-!\cA-!\cB-!\cA-!\cB-!\cA-!\cB-!\cA-\phantom{Y}
      } &
      \\
      Random (statisztikus) & \chemfig[atom sep=2em]{
        \phantom{X}-!\cA-!\cA-!\cA-!\cB-!\cA-!\cB-!\cB-\phantom{Y}
      } & RPP, EVA, SBR
      \\
      Blokk                 & \chemfig[atom sep=2em]{
        \phantom{X}-!\cA-!\cA-\dots-!\cA-!\cB-\dots-!\cB-\phantom{Y}
      } & SB, SBS
      \\
      Ojtott                & \chemfig[atom sep=2em]{
        \phantom{X}-!\cA-!\cA-!\cA
        (-[::-130]!\cB-[::-50]!\cB-[::+0]\dots)
        -\dots-!\cA
        (-[::-50]!\cB-[::+50]!\cB-[::+0]\dots)
        -!\cA-!\cA-\phantom{Y}
      } & HiPS, ABS
    \end{tabular}
  \end{table}
  % indent



  \tcbitem[title={\# \thetcbrasternum{} –
        Mi a kovalens kötés?
      }]
  Polimereknél nagy jelentőségű \textbf{elsőrendű} (kémiai, intermolekuláris)
  kötés. A kapcsolódó atomok megosztják a vegyértékelektronjaikat.
  \begin{center}
    \chemfig[atom sep=2em]{H-H}
    \hspace{2em}
    \chemfig[atom sep=2em]{\charge{135=\|,225=\|}{O}=\charge{45=\|,-45=\|}{O}}
    \hspace{2em}
    \chemfig[atom sep=2em]{\charge{180=\|}{C}~\charge{0=\|}{O}}
  \end{center}



  \tcbitem[title={\# \thetcbrasternum{} –
        Mit jelent a polimerizációs fok?
      }]
  A polimerizációs fok (\textbf{DP}, \emph{degree of polymerization}) azt
  fejezi ki, hogy hány monomerből polimerizáltuk az adott molekulaláncot.



  \tcbitem[title={\# \thetcbrasternum{} –
        Sorolja fel a tömegműanyagokat és rajzolja fel az ismétlő egységeiket.
      }]
  % noindent
  % \begin{multicols}{3}
  % PE
  \begin{figure}[H]
    \centering
      \chemfig[
        atom sep=2em,
        bond style={line width=1pt,red}
      ]{\vphantom{X}-[@{op}]C(-[2]H)(-[6]H)-C(-[@{cp}])(-[2]H)(-[6]H)-\vphantom{X}}
      \polymerdelim[
        delimiters ={[]},
        height = 2.5em,
        depth = 2.5em,
        indice = n
      ]{op}{cp}
    \caption{Polietilén}
  \end{figure} 

  % PP
  \begin{figure}[H]
    \centering
    \chemfig[
      atom sep=2em,
      bond style={line width=1pt,red}
    ]{
      \vphantom{X}
      -[@{op,.25}, 1.25]C(-[::+120]H)(-[::+240]H)
      -C(-[::+60]H)
      (-[6]C(-[6]H)(-[::-60]H)(-[::+60]H))
      -[@{cp,.875},1.5]\vphantom{X}}
    \polymerdelim[
      delimiters ={[]},
      height = 2.5em,
      depth = 4.5em,
      indice = n
    ]{op}{cp}
    \caption{Polipropilén}
  \end{figure}
  
  % PVC
  \begin{figure}[H]
    \centering
    \chemfig[
      atom sep=2em,
      bond style={line width=1pt,red}
    ]{\vphantom{X}-[@{op}]C(-[2]H)(-[6]H)-C(-[@{cp}])(-[2]H)(-[6]Cl)-\vphantom{X}}
    \polymerdelim[
      delimiters ={[]},
      height = 2.5em,
      depth = 2.5em,
      indice = n
    ]{op}{cp} 
    \caption{Polivinil-klorid}
  \end{figure}
    
  % PET
  \begin{figure}[H]
    \centering
    \chemfig[
      atom sep=2em,
      bond style={line width=1pt,red}
    ]{
      \vphantom{X}
      -[@{op,0.25}, 1.25]C(-[::+120]H)(-[::+240]H)
      -C(-[::+60]H)(-[::-60]H)
      -O
      -[::-60]C(=[::-60]O)
      -*6([,.75]-=-([,1]-C(=[::-60]O)-[::+60]O(-[@{cp}])-\vphantom{X})=-=)
    }
    \polymerdelim[
      delimiters ={[]},
      height = 2.5em,
      depth = 4.5em,
      indice = n
    ]{op}{cp}
    \caption{Polietilén-tereftalát}
  \end{figure}

  % PS
  \begin{figure}[H]
    \centering
    \chemfig[
      atom sep=2em,
      bond style={line width=1pt,red}
    ]{
      \vphantom{X}
      -[@{op,.25}, 1.25]C(-[::+120]H)(-[::+240]H)
      -C(-[@{cp,.75}, 1.25])(-[::+60]H)(-[6]*6([6,.75]-=-=-=))
      -\vphantom{X}
    }
    \polymerdelim[
      delimiters ={[]},
      height = 2.5em,
      depth = 5.25em,
      indice = n
    ]{op}{cp}
    \caption{Polisztirol}
  \end{figure}
  % \end{multicols}
  % indent



  \tcbitem[title={\# \thetcbrasternum{} –
        Mi a fázis (anyagtudomány)?
      }]
  A fázis az anyag olyan \textbf{lényegi része}, amelynek kémiai összetétele és
  fizikai állapota megegyezik (\textbf{egységes}). A \textbf{fázisdomén} egy
  összefüggő területe az anyagnak.



  \tcbitem[title={\# \thetcbrasternum{} –
        Mit jelent polimerek esetében a szegmens?
      }]
  A szegmens a polimermolekula \textbf{termodinamikailag} összefüggő része, a
  molekulalánc kisebb szakasza. A szegmensek együtt mozognak. A
  szegmensmozgások (Mikro-Brown mozgások) esetén a molekula egyes részei
  elmozdulnak, konformációs változások jöhetnek létre. Méretük leggyakrabban
  néhány ismétlődő egység.



  \tcbitem[title={\# \thetcbrasternum{} –
        Mit jelent az, hogy egy polimer amorf?
      }]
  Az amorf anyagok \textbf{egyetlen, amorf fázisból} állnak. A molekulák
  tartalmazhatnak elágazásokat, oldalláncokat. A molekulák itt
  \textbf{rendezetlenül} helyezkednek el a térben, azaz a \textbf{molekulák
    közötti távolság nem állandó}. Az amorf anyagok lehetnek
  \textbf{lineárisak} és \textbf{térhálósak} is.



  \tcbitem[title={\# \thetcbrasternum{} –
        Mit jelent az, hogy egy polimer részben kristályos?
      }]
  Részben kristályos anyagoknál az a jellemző, hogy a felhasználás
  hőmérsékletén, szilárd halmazállapotban az \textbf{amorf részek mellett
    térben rendezett részek is} jelentős mértékben találhatóak bennük. Ezek a
  polimerek \textbf{lineáris szerkezetűek}. \textbf{Szabályosság, állandó
    molekulatávolság} figyelhető meg. Mindig van benne amorf rész, nincsen
  teljesen kristályos polimer.



  \tcbitem[title={\# \thetcbrasternum{} –
        Soroljon fel 3 részben-kristályos és 3 amorf polimer típust!
      }]
  \setlength{\columnsep}{-1cm}
  \begin{multicols}{2}
    \begin{itemize}
      \item Részben kristályos:
            \begin{itemize}
              \item PET (Polietilén-tereftalát)
              \item PE (Polietilén)
              \item PP (Polipropilén)
              \item PA (Poliamid)
              \item PTFE (Teflon)
            \end{itemize}
      \item Amorf:
            \begin{itemize}
              \item PVC (Polivinil-klorid)
              \item PS (Polisztirol)
              \item ABS (Akrilnitril-butadién-sztirol)
              \item PC (Polikarbonát)
              \item PMMA (Polimetil-metakrilát)
            \end{itemize}
    \end{itemize}
  \end{multicols}



  \tcbitem[title={\# \thetcbrasternum{} –
        Mi a szferolit?
      }]
  A szferolit mikroszkóp alatt látható \textbf{szuperkristály}. Gömbszerű
  szerkezet, nem csak polimerekben figyelhető meg. Az anyagban az
  \textbf{inhomogenitás} (pl. szennyeződés) környékén indul meg a lamellák
  (\textbf{krisztallitok}) kialakulása. Kisszögű elágazások mentén lévő
  lamellacsoportosulásokat \textbf{fibrilláknak} nevezzük. Lehet sünis és kévés
  szerkezetű.



  \tcbitem[title={\# \thetcbrasternum{} –
        Hogyan számítható a polimerek húzószilárdsága?
      }]
  A húzószilárdság a szakítógörbe \textbf{első lokális maximumánál} ébredő
  mérnöki feszültség.



  \tcbitem[title={\# \thetcbrasternum{} –
        Írja fel a Hooke-törvényt (szilárd testekre vonatkozólag)!
      }]
  A Hooke-törvény az feszültég és az alakváltozás közötti egyenes arányosságot
  fejezi ki.
  \[
    \sigma = E \cdot \varepsilon
  \]



  \tcbitem[title={\# \thetcbrasternum{} –
        Hogyan határozhatjuk meg egy polimer húzási rugalmassági modulusát?
      }]
  A húzási rugalmassági modulus a \textbf{szakítógörbe meredeksége}. A
  feszültség és az alakváltozás közötti kapcsolatot fejezi ki. Megmutatja, hogy
  adott terhelésre milyen nyúlással reagál az anyag. Fajtái:
  % \begin{multicols}{3}
  \begin{itemize}
    \item kezdeti rug. mod.
    \item húr mod.
    \item érintő mod.
  \end{itemize}
  % \end{multicols}



  \tcbitem[title={\# \thetcbrasternum{} –
        Mit jelent az üveges átmeneti hőmérséklet-tartomány?
      }]
  Az üvegesedési átmeneti hőmérséklet tartományában a polimer anyag
  \textbf{üvegesen amorf} állapotból \textbf{nagyrugalmas amorf} állapotba lép
  át.



  \tcbitem[title={\# \thetcbrasternum{} –
        Mit jelent az, hogy egy polimer viszkoelasztikus?
      }]
  Viszkoelasztikus anyagoknál \textbf{rugalmas} (elasztikus) és
  \textbf{viszkózus} viselkedés is megfigyelhető. A feszültség-deformáció
  kapcsolat \textbf{nemlineáris}, a tulajdonságok a terhelési szinttől,
  terhelés időtartamától és a hőmérséklettől függenek. Összetett viselkedésüket
  ideális tulajdonságok kombinációjaként írjuk le. Az összdeformáció 3
  komponensből tevődik össze:
  \begin{itemize}
    \item $\varepsilon_{pr}$ – \textbf{pillanatnyi rugalmas} komponens
          \begin{itemize}
            \item Hooke-törvényt ($\sigma = E \cdot \varepsilon$) követő
                  ideális rugó.
            \item Atomtávolságok és vegyértékszögek megváltozása.
            \item Mechanikailag és termodinamikailag is reverzibilis.
          \end{itemize}
    \item $\varepsilon_{m}$ – \textbf{maradó} deformáció
          \begin{itemize}
            \item Newton-törvényt ($\sigma = \eta \cdot \dot{\varepsilon}$)
                  követő ideális viszkózus elem.
            \item Molekulaláncok egymáshoz képest elcsúsznak.
            \item Mechanikailag és termodinamikailag is irreverzibilis.
          \end{itemize}
    \item $\varepsilon_{kr}$ – \textbf{késleltetett rugalmas} komponens
          \begin{itemize}
            \item Kelvin-Voigt elem (rugó és viszkózus elem páthuzamos
                  kapcsolása).
            \item Molekulaláncok ki- és visszagöngyölődése.
            \item Szegmensmozgások, konformációváltozások
            \item Mechanikailag reverzibilis, de termodinamikailag nem.
            \item Hiszterézishurok jellemzi.
          \end{itemize}
  \end{itemize}
  A viselkedés a \textbf{Burgers-féle négyparaméteres modellel} írható le.



  \tcbitem[title={\# \thetcbrasternum{} –
        Mi a kúszás?
      }]
  A kúszás (nyúlásrelaxáció) vizsgálat egy \textbf{statikus} vizsgálat, ahol a
  próbatestre ugrásszerűen egy bizonyos időre \textbf{állandó nagyságú
    feszülséget} kapcsolunk. Megfigyelhetjük, hogy a próbatest
  \textbf{alakváltozása} az idő függvényében \textbf{folyamatosan nő}.



  \tcbitem[title={\# \thetcbrasternum{} –
        Mi a feszültségrelaxáció?
      }]
  A feszültségrelaxáció vizsgálat szintén egy \textbf{statikus} vizsgálat,
  hiszen itt is állandó a gerjesztés. Megfigyelhetjük, hogy \textbf{állandó
    alakváltozás} fenntartásához \textbf{egyre kisebb húzófeszültségre} van
  szükség.




  \tcbitem[title={\# \thetcbrasternum{} –
        Mi a DM(T)A?
      }]
  A DMA (dinamikus mechanikai analízis) egy \textbf{fárasztóvizsgálat}. Egy $4
    \cross 10 \cross 60 \, \mathrm{mm}$ méretű hasábot \textbf{ciklikusan}
  (szinuszosan) \textbf{terheljük} és a feszültséget, alakváltozást az idő
  függvényében vizsgáljuk. A próbatestet csavaró vagy nyíró igénybevétellel is
  terhelhetjük. Az alakváltozás és a feszültség között fáziskésés figyelhető
  meg:
  \begin{align*}
    \varepsilon(t) & = \varepsilon_0 \cdot \sin(\omega t)
    \\
    \sigma(t)      & = \sigma_0 \cdot \sin(\omega t + \delta)
    \\
                   & = \sigma_0 \left(
    \cos \delta \sin(\omega t) +
    \sin \delta \cos(\omega t)
    \right)
    \\
                   & = \sigma'_0 \sin(\omega t) + \sigma''_0 \cos(\omega t)
  \end{align*}



  \tcbitem[title={\# \thetcbrasternum{} –
        Definiálja a viszkozitást!
      }]
  A viszkozitás egy közeg ellenállásának mértéke a csúsztatófeszültség okozta
  alakváltozással szemben. Egy közeg \textbf{belső súrlódásaként} is
  felfogható. Nagyobb viszkozitású anyag sűrűbben folyik.



  \tcbitem[title={\# \thetcbrasternum{} –
        Írja fel a Newton törvényt folyadékok esetére!
      }]
  A Newton-törvényt követő ideálisan viszkózus folyadékkal töltött dugattyús
  henger:
  \[
    \sigma = \eta \cdot \dot{\varepsilon}
  \]
  Newton-törvény folyadékokra:
  \[
    \tau = \eta \cdot \dot{\gamma}
  \]
  $\tan{\gamma} = \gamma$ közelítéssel élve:
  \[
    \dot{\gamma}
    = \lim_{\Delta t \rightarrow 0} \frac{\Delta \gamma}{\Delta t}
    = \lim_{\Delta t \rightarrow 0} \frac{\Delta l}{\Delta t \cdot d}
    = \frac{v}{d}
  \]
  Ahol $v$ a felső lap sebessége, $\dot{\gamma}$ a deformáció
  sebessége/nyírósebesség. (?sebességgradiensnek is szokás nevezni?)



  \tcbitem[title={\# \thetcbrasternum{} –
        Mi az MFI?
      }]
  Az MFI (Melt flow index) a \textbf{folyóképesség} gyakorlati jellemzésére
  használt \textbf{szabványos folyási mutatószám}, megadja azt a grammokban
  kifejezett anyagmennyiséget, amely a vizsgálati és anyagszabványban előírt
  hőmérséklet és nyomás mellett a szabványos mérőkészülék kifolyónyílásán
  10 perc alatt kifolyik.



  \tcbitem[title={\# \thetcbrasternum{} –
        Mi a hengerszék?
      }]
  A hengerszék eljárás \textbf{nagy viszkozitású} anyagokat és adalékanyagokat
  kever. \textbf{Szakaszos üzemű, nyitott} eljárás. \textbf{Két} azonos
  átmérőjű, temperált \textbf{fémhenger} különböző sebességgel egymással
  szemben forog. Az anyag az érdesebb, gyorsabban forgó, melegebb hengerre
  tapad. \textbf{Szakállképződés} jellemző, amely az anyag feltorlódását
  jelenti. A frikció a két henger szögsebességének hányadosa ($f = \omega_1 /
    \omega_2 \approx 1,1 \dots 1,4$). Nagy emberi tényező: veszélyes,
  \textbf{kézi etetés}, tapasztalat szükséges. Jó homogenitású keverék hozható
  létre. Egyszerű karbantartás, tisztitás, hőmérséklet pontosan beállíthtó.



  \tcbitem[title={\# \thetcbrasternum{} –
        Mi a kalander?
      }]
  A kalanderezés a \textbf{hengerszék} tenchnológiájából alakult ki. 3, 4, 5
  hengerből álló \textbf{hengersor}, mellyel \textbf{folytonos gyártás}
  valósítható meg. Az eljárással \textbf{fóliákat}, vékony \textbf{lemezeket}
  gyárthatunk. A hengerek felülete tükrös, edzett, nitridált. A hengerek
  temperáltak, egymással ellentétes irányba forognak. Különböző elrendezések
  valósíthatóak meg (\textit{I, L, F, Z, \dots}).



  \tcbitem[title={\# \thetcbrasternum{} –
        Mi az extrúzió?
      }]
  Tipikusan \textbf{termoplasztikus} (hőre lágyuló) polimert az extrúder
  \textbf{képlékeny állapotba} hozza, majd a viszkózusan folyós ömledéket
  \textbf{komprimálja} (nyomás alá helyezi), \textbf{homogenizálja}, adott,
  változatlan keresztmetszetű, nyitott szerszámon \textbf{keresztülsajtolja},
  méretállandóságot követőberendezésekkel biztosítva \textbf{lehűti}, és így
  \textbf{állandó keresztmetszetű} polimer terméket gyárt tetszőleges
  hosszúságban, \textbf{folytonos üzemben}. A kijövő termék kiterjedése tehát
  az egyik dimenzióban végtelen, ami lehet cső, síklap, profilos hasáb, fólia,
  stb. Az alapanyag por vagy granulátum, adalékanyagokkal.



  \tcbitem[title={\# \thetcbrasternum{} –
        Mi a fröccsöntés?
      }]
  Polimer késztermékek előállítására alkalmas szakaszos (\textbf{ciklikus})
  \textbf{eljárás}. Alapelve, hogy a \textbf{termoplasztikus} (hőre lágyuló)
  polimer alapanyagot olvadáspontja fölé, vagyis \textit{viszkózusan folyós}
  \textbf{ömledékállapotba} hozzuk, majd ezt \textbf{nagy sebességgel} és
  \textbf{nyomással}, szűk beömlőnyíláson át egy zárt, \textbf{temperált}
  (szabályozott hőmérsékletű) \textbf{szerszámba} juttatjuk. \textbf{
    Tetszőlegesen bonyolult}, 3D-s, nagy méretpontosságú alkatrész
  alakítható ki gyakolatilag \textbf{hulladékmentesen}.



  \tcbitem[title={\# \thetcbrasternum{} –
        Mi a kompozit?
      }]
  A kompozit \textbf{többfázisú} (alkotóiban fázishatárokkal elválasztott),
  több alkotóból álló \textbf{összetett} szerkezeti anyag, amely
  \textbf{erősítőanyagból} (szálerősítő) és \textbf{befoglaló} mátrixanyagból
  áll, és az jellemzi, hogy a \textbf{nagy szilárdságú} és \textbf{nagy
    rugalmassági modolusú} (szálas) \textbf{erősítőanyag} és a rendszerint
  \textbf{kisebb szilárdságú mátrix} között kitűnő első vagy másodrendű kötések
  általi \textbf{adhéziós} kapcsolat van, amely a deformáció magas szintjén is
  \textbf{tartósan fennmarad}. A kompozitok kialakítása abból a felismerésből
  jött létre, hogy az alkatrészek terhelésének iránya meghatározható, ebbe az
  irányba nagyobb szilárdságra van szükség.

\end{tcbitemize}

\end{document}
